\documentclass{beamer}
\usetheme{Warsaw}

\usepackage{graphicx} % Allows including images
\usepackage{booktabs} % Allows the use of \toprule, \midrule and \bottomrule in tables
\usepackage{listings}
\usepackage[utf8]{inputenc}
\usepackage[overlay,absolute]{textpos}
\usepackage[]{algorithm2e}
\usepackage{amssymb}
\usepackage{tikz}
\usetikzlibrary{arrows, automata}

\AtBeginSection[]
{
\begin{frame}<beamer>
\frametitle{Plan}
\tableofcontents[
  currentsection,
  hideothersubsections
]
\end{frame}
}

\lstset{language=C++,
                basicstyle=\ttfamily,
                keywordstyle=\color{green}\ttfamily,
                stringstyle=\color{red}\ttfamily,
                commentstyle=\color{cyan}\ttfamily,
                morecomment=[l][\color{magenta}]{\#}
}

\setbeamercolor{normal text}{fg=white,bg=black!90}
\setbeamercolor{structure}{fg=white}

\setbeamercolor{alerted text}{fg=red!85!black}

\setbeamercolor{item projected}{use=item,fg=black,bg=item.fg!35}

\setbeamercolor*{palette primary}{use=structure,fg=structure.fg}
\setbeamercolor*{palette secondary}{use=structure,fg=structure.fg!95!black}
\setbeamercolor*{palette tertiary}{use=structure,fg=structure.fg!90!black}
\setbeamercolor*{palette quaternary}{use=structure,fg=structure.fg!95!black,bg=black!80}

\setbeamercolor*{framesubtitle}{fg=white}

\setbeamercolor*{block title}{parent=structure,bg=black!60}
\setbeamercolor*{block body}{fg=black,bg=black!10}
\setbeamercolor*{block title alerted}{parent=alerted text,bg=black!15}
\setbeamercolor*{block title example}{parent=example text,bg=black!15}

\author[Félix-Antoine Ouellet]{Félix-Antoine Ouellet}

\title[PolyOpt\hspace{2em}\insertframenumber/\inserttotalframenumber]{Compilation polyhédrale}

\institute{Université de Sherbrooke}

\date{4 décembre 2014}

\begin{document}

\begin{frame}
\titlepage % Print the title page as the first slide
\end{frame}

\begin{frame}
\tableofcontents[hideallsubsections]
\end{frame}

\section{Motivation}
\begin{frame}
\frametitle{L'ère du parallélisme}

\end{frame}

\begin{frame}
\frametitle{Problèmes courants}
\framesubtitle{Optimisations des boucles}

\end{frame}

\begin{frame}
\frametitle{Problèmes courants}
\framesubtitle{Parallélisation d'applications existantes}

\end{frame}

\begin{frame}
\frametitle{Problèmes courants}
\framesubtitle{Rendre le parallélisme accessible}

\end{frame}

\section{Compilation traditionnelle}
\subsection{Bases de la compilation}
\begin{frame}
\frametitle{Notions importantes}
\begin{itemize}
\item Transforme un programme écrit dans un langage (de haut niveau) en un programme écrit dans un autre langage (de bas niveau).
\item Maintient la sémantique du programme original.
\end{itemize}
\end{frame}

\begin{frame}
\frametitle{Architecture usuelle}

\end{frame}

\subsection{Processus de compilation}
\begin{frame}
\frametitle{Étape 1 - AST}

\end{frame}

\begin{frame}
\frametitle{Étape 2 - CFG}

\end{frame}

\begin{frame}
\frametitle{Représentation intermédiaire}
POURQUOI ??
\begin{itemize}
\item 
\end{itemize}
\end{frame}

\subsection{Analyse de dépendences}
\begin{frame}
\frametitle{Importance des dépendences}

\end{frame}

\begin{frame}
\frametitle{Représentation}

\end{frame}

\begin{frame}
\frametitle{Tests de dépendences}

\end{frame}

\section{Approche polyhédrale}
\subsection{Représentation}
\begin{frame}
\frametitle{Représentation}

\end{frame}

\subsection{Optimisations}

\subsection{Limitations}
\begin{frame}
\frametitle{Limitations}
\begin{itemize}
\item Accès non affines
\item Boucles irrégulières
\end{itemize}
\end{frame}

\section{Parallélisation automatique}
\begin{frame}
\frametitle{Intuition}

\end{frame}

\subsection{Schedule}

\subsection{Support présent}
\begin{frame}
\frametitle{Support présent}
\begin{itemize}
\item GCC (Graphite)
\item LLVM (Polly)
\item Langages expérimentaux (X10)
\item Plateformes expérimentales (PLUTO)
\end{itemize}
\end{frame}

\section{Conclusion}
\begin{frame}
\frametitle{Conclusion}
\end{frame}

\end{document}