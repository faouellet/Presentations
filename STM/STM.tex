\documentclass{beamer}
\usetheme{Warsaw}

\usepackage{graphicx} % Allows including images
\usepackage{booktabs} % Allows the use of \toprule, \midrule and \bottomrule in tables
\usepackage{listings}
\usepackage[utf8]{inputenc}
\usepackage[overlay,absolute]{textpos}
\usepackage{adjustbox}
\usepackage[]{algorithm2e}
\usepackage{colortbl}
\usepackage{amsfonts}

\AtBeginSection[]
{
\begin{frame}<beamer>
\frametitle{Plan}
\tableofcontents[
  currentsection,
  hideothersubsections
]
\end{frame}
}

\lstset{language=C++,
        basicstyle=\ttfamily,
        keywordstyle=\color{green}\ttfamily,
        stringstyle=\color{red}\ttfamily,
        commentstyle=\color{cyan}\ttfamily,
        morecomment=[l][\color{magenta}]{\#},
        escapechar=@
}

\definecolor{DGreen}{rgb}{0,.8,0}

\setbeamercolor{normal text}{fg=white,bg=black!90}
\setbeamercolor{structure}{fg=white}

\setbeamercolor{alerted text}{fg=red!85!black}

\setbeamercolor{item projected}{use=item,fg=black,bg=item.fg!35}

\setbeamercolor*{palette primary}{use=structure,fg=structure.fg}
\setbeamercolor*{palette secondary}{use=structure,fg=structure.fg!95!black}
\setbeamercolor*{palette tertiary}{use=structure,fg=structure.fg!90!black}
\setbeamercolor*{palette quaternary}{use=structure,fg=structure.fg!95!black,bg=black!80}

\setbeamercolor*{framesubtitle}{fg=white}

\setbeamercolor*{block title}{parent=structure,bg=black!60}
\setbeamercolor*{block body}{fg=black,bg=black!10}
\setbeamercolor*{block title alerted}{parent=alerted text,bg=black!15}
\setbeamercolor*{block title example}{parent=example text,bg=black!15}

\author[Félix-Antoine Ouellet]{Félix-Antoine Ouellet}

\title[MTL\hspace{2em}\insertframenumber/\inserttotalframenumber]{Mémoire transactionnelle logicielle}

\institute{Université de Sherbrooke}

\date{23 octobre 2014}

\begin{document}

\begin{frame}
\titlepage % Print the title page as the first slide
\end{frame}

\begin{frame}
\tableofcontents[hideallsubsections]
\end{frame}

\section{Motivation}
\subsection{Programmation parallèle traditionnelle}

\newlength\someheight
\setlength\someheight{3.5cm}

\begin{frame}[fragile]
\frametitle{Programmation parallèle traditionnelle}
\framesubtitle{Exemple}
\begin{adjustbox}{width=\textwidth,height=\someheight,keepaspectratio}
\begin{lstlisting}
class Account {
private:
  int m_Balance;
  std::mutex m_Mutex;
  
public:
  void deposit(int n) {
    m_Mutex.lock();
    m_Balance += n;
    m_Mutex.unlock();
  }

  void withdraw(int n) {
    m_Mutex.lock();
    if (m_Balance >= n) {
      m_Balance -= n;
    }
    m_Mutex.unlock();
  }
};
\end{lstlisting}
\end{adjustbox}
\end{frame}

\begin{frame}[fragile]
\frametitle{Programmation parallèle traditionnelle}
\framesubtitle{Exemple}
\begin{lstlisting}
void transfer(Account & From, Account & To, 
              int Amount) {
  From.withdraw(Amount);
  To.deposit(Amount);
}
\end{lstlisting}
\end{frame}

\begin{frame}[fragile]
\frametitle{Programmation parallèle traditionnelle}
\framesubtitle{Exemple}
\begin{lstlisting}
void transfer(Account & From, Account & To, 
              int Amount) {
  From.lock();
  To.lock();
  From.withdraw(Amount);
  To.deposit(Amount);
  From.unlock();
  To.unlock();
}
\end{lstlisting}
\end{frame}

\subsection{Problèmes}
\begin{frame}
\frametitle{Problèmes}
\begin{itemize}
\item Problèmes de synchronisation
\item Difficile à composer
\item Penser parallèle
\end{itemize}
\end{frame}

\section{Mémoire transactionnelle}
\subsection{Concept}
\begin{frame}
\frametitle{Concept}
\begin{itemize}
\item Suite d'instructions exécutées d'une manière semblable aux transactions dans une base de données
\begin{itemize}
\item<2-> Atomique: Aucun état intermédiaire visible
\item<3-> Isolée: Non affectée par les autres \textit{threads}
\end{itemize}
\end{itemize}
\end{frame}

\begin{frame}[fragile]
\frametitle{Concept}
\framesubtitle{Exemple}
\begin{lstlisting}
void transfer(Account & From, Account & To, 
              int Amount) {
  atomic {
    From.withdraw(Amount);
    To.deposit(Amount);
  }
}
\end{lstlisting}
\end{frame}

\subsection{Possible implémentation}
\begin{frame}
\frametitle{Possible implémentation}
\begin{algorithm}[H]
 Créer un \textit{log} local au \textit{thread} exécutant la section atomique\;
 \Repeat{Transaction valide}{
  Exécuter instructions sur une copie courante des variables\;
  Valider transaction (atomiquement)\;
  \If{Transaction valide}{
  Écrire transaction (atomiquement)\;
   }
 }
\end{algorithm}
\end{frame}

\begin{frame}[fragile]
\frametitle{Possible implémentation}
\framesubtitle{Exemple}
\begin{columns}
    \begin{column}{0.48\textwidth}
        \begin{lstlisting}
From::m_Balance = 10;
To::m_Balance = 10;       
        
atomic {
  From.withdraw(10);
  To.deposit(10);
}
\end{lstlisting}
    \end{column}
    \begin{column}{0.55\textwidth}
        \begin{tabular}{ | c | c | c | }
  \hline
  Addresse & Valeur & Type \\
  \hline
  \onslide<2->{\&From::m\_Balance & 10 & L} \\
  \hline
  \onslide<3->{\&From::m\_Balance & 0 & É} \\
  \hline
  \onslide<4->{\&To::m\_Balance & 10 & L} \\
  \hline
  \onslide<5->{\&To::m\_Balance & 20 & É} \\
  \hline
\end{tabular}
    \end{column}
\end{columns}
\end{frame}

\begin{frame}[fragile]
\frametitle{Possible implémentation}
\framesubtitle{Exemple}
\begin{columns}
    \begin{column}{0.48\textwidth}
        \begin{lstlisting}
From::m_Balance = @\textcolor{green}{10}@;
To::m_Balance = @\textcolor{green}{10}@;       
        
atomic {
  From.withdraw(10);
  To.deposit(10);
}
\end{lstlisting}
    \end{column}
    \begin{column}{0.55\textwidth}
        \begin{tabular}{ | c | c | c | }
  \hline
  Addresse & Valeur & Type \\
  \hline
  \rowcolor{DGreen}\&From::m\_Balance & 10 & L \\
  \hline
  \&From::m\_Balance & 0 & É \\
  \hline
  \rowcolor{DGreen}\&To::m\_Balance & 10 & L \\
  \hline
  \&To::m\_Balance & 20 & É \\
  \hline
\end{tabular}
    \end{column}
\end{columns}
\end{frame}

\begin{frame}[fragile]
\frametitle{Possible implémentation}
\framesubtitle{Exemple}
\begin{columns}
    \begin{column}{0.48\textwidth}
        \begin{lstlisting}
From::m_Balance = @\textcolor{green}{10}@;
To::m_Balance = @\textcolor{red}{20}@;       
        
atomic {
  From.withdraw(10);
  To.deposit(10);
}
\end{lstlisting}
    \end{column}
    \begin{column}{0.55\textwidth}
        \begin{tabular}{ | c | c | c | }
  \hline
  Addresse & Valeur & Type \\
  \hline
  \rowcolor{DGreen}\&From::m\_Balance & 10 & L \\
  \hline
  \&From::m\_Balance & 0 & É \\
  \hline
  \rowcolor{red}\&To::m\_Balance & 10 & L \\
  \hline
  \&To::m\_Balance & 20 & É \\
  \hline
\end{tabular}
    \end{column}
\end{columns}
\end{frame}

\section{Évaluation}
\subsection{Avantages}
\begin{frame}[fragile]
\frametitle{Avantages}
\begin{adjustbox}{width=\textwidth,height=\someheight,keepaspectratio}
\begin{lstlisting}
void transfer(Account & From, Account & To, 
              int Amount) {
  atomic {
    From.withdraw(Amount);
    To.deposit(Amount);
  }
}
\end{lstlisting}
\end{adjustbox}
\end{frame}

\begin{frame}[fragile]
\frametitle{Avantages}
\begin{adjustbox}{width=\textwidth,height=\someheight,keepaspectratio}
\begin{lstlisting}
void transfer(Account & From, Account & To, 
              int Amount) {
  atomic {
    From.withdraw(Amount);
    To.deposit(Amount);
  }
}
/*...*/
atomic {
  transfer(Account1, Account2, 500);
  transfer(Account2, Account3, 300);
}
\end{lstlisting}
\end{adjustbox}

\begin{textblock}{5}(10, 8)
	 \begin{itemize}
	 \item Composable
	 \item<2-> Évite les \textit{deadlocks}
	 \end{itemize}
\end{textblock}
\end{frame}

\begin{frame}[fragile]
\frametitle{Avantages}
\framesubtitle{Facilite le raisonnement}
\begin{columns}
    \begin{column}{0.5\textwidth}
    Niveau professionnel
    \begin{adjustbox}{width=\textwidth,height=\someheight,keepaspectratio}
        \begin{lstlisting}  
void transfer(Account & From, 
              Account & To, 
              int Amount) {
  From.lock();
  To.lock();
  From.withdraw(Amount);
  To.deposit(Amount);
  From.unlock();
  To.unlock();
}
        \end{lstlisting}
        \end{adjustbox}
    \end{column}
    \begin{column}{0.5\textwidth}
    Niveau étudiant
    \begin{adjustbox}{width=\textwidth,height=\someheight,keepaspectratio}
        \begin{lstlisting}        
void transfer(Account & From, 
              Account & To, 
              int Amount) {
  atomic {
    From.withdraw(Amount);
    To.deposit(Amount);
  }
}
        \end{lstlisting}
        \end{adjustbox}
    \end{column}
\end{columns}
\end{frame}

\subsection{Problèmes potentiels}
\begin{frame}
\frametitle{Problèmes potentiels}
\framesubtitle{Problèmes d'implémentation et de performance}
\begin{itemize}
\item Solution avec verrous: Comment être performant?
\item Solution sans verrous: Voir présentation précédente
\item Solution hybride: Requiert beaucoup de finesse
\end{itemize}
\end{frame}

\begin{frame}[fragile]
\frametitle{Problèmes potentiels}
\framesubtitle{Interactions avec code non-transactionnel}
\begin{center}
// Initialement: x\_b = true;
\end{center}
\begin{columns}
    \begin{column}{0.5\textwidth}
    Thread 1
        \begin{lstlisting}        
atomic {
  x_b = false;
}
x_i = 100;
        \end{lstlisting}
    \end{column}
    \begin{column}{0.5\textwidth}
    Thread 2
        \begin{lstlisting}        
atomic {
  if (x_b) {
    x_i = 1;
  }
}
        \end{lstlisting}
    \end{column}
\end{columns}
\begin{center}
Que vaut x à la fin du programme?
\end{center}
\end{frame}

\begin{frame}[fragile]
\frametitle{Problèmes potentiels}
\framesubtitle{Gestions des exceptions}
\begin{lstlisting}
void fonction() {
  /*...*/
  atomic {
    i++;
    throw 42;
  }
  /*...*/
}
\end{lstlisting}

\begin{textblock}{5}(9, 8)
	 \huge{Que faire?}
\end{textblock}
\end{frame}

\section{Support présent}
\subsection{Langages de programmation}
\begin{frame}[fragile]
\frametitle{C++}
Extension disponible dans GCC depuis GCC 4.7
\begin{lstlisting}
void transfer(Account & From, Account & To, 
              int Amount) {
  __transaction_atomic {
    From.withdraw(Amount);
    To.deposit(Amount);
  }
}
\end{lstlisting}
\end{frame}

\begin{frame}[fragile]
\frametitle{C++}
En voie d'être standardisé pour C++17
\begin{lstlisting}
void transfer(Account & From, Account & To, 
              int Amount) {
  atomic_noexcept {
    From.withdraw(Amount);
    To.deposit(Amount);
  }
}
\end{lstlisting}
\end{frame}

\subsection{Matériel}
\begin{frame}
\frametitle{Matériel}
\begin{center}
\colorbox{white}{\includegraphics[scale=0.15]{haswell.jpg}}
\end{center}
\end{frame}

\section{Conclusion}
\begin{frame}
\frametitle{Conclusion}
\begin{itemize}
\item Avenue intéressante pour le futur de la programmation parallèle
\item<2-> Loin d'être arriver à maturité
\end{itemize}
\end{frame}

\end{document}